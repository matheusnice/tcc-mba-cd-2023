%% USPSC-Abstract.tex
%\autor{Silva, M. J.}
\begin{resumo}[Abstract]
 \begin{otherlanguage*}{english}
	\begin{flushleft} 
		\setlength{\absparsep}{0pt} % ajusta o espaçamento dos parágrafos do resumo		
 		\SingleSpacing  		\imprimirautorabr~~\textbf{\imprimirtitleabstract}.	\imprimirdata.  \pageref{LastPage}p. 
		%Substitua p. por f. quando utilizar oneside em \documentclass
		%\pageref{LastPage}f.
		\imprimirtipotrabalhoabs~-~\imprimirinstituicao, \imprimirlocal, 	\imprimirdata. 
 	\end{flushleft}
	\OnehalfSpacing 
	This work aims to apply machine learning models and explainable artificial intelligence techniques to the data from the Coordination of Hospital Admissions in Ribeirão Preto, Brazil, with the goal of predicting hospital readmissions. The methodology encompassed the training and selection of classifiers in three distinct phases, culminating in the analysis of variable importance through the SHAP method. Results indicate that unmarried patients who had shorter hospital stays and are younger present a higher risk of readmission. Seasonal variables, such as the month or day of the week of admission, also showed significant influence on readmission prediction. However, model accuracies remained below 60\%, pointing to opportunities for improvement in predictive performance. Alternatives were suggested for future enhancements, such as expanding the dataset and exploring more complex models. These conclusions, in addition to guiding specific improvements in this context, provide a robust foundation for future research, not only in this specific dataset but as guidelines for similar investigations in other datasets, contributing to ongoing advancements in predicting hospital readmissions.
	
	\vspace{\onelineskip}
	\noindent 
	\textbf{Keywords}: Explainable Artificial Intelligence. Machine Learning. Hospital Readmissions. Mental Health Services.
 \end{otherlanguage*}
\end{resumo}