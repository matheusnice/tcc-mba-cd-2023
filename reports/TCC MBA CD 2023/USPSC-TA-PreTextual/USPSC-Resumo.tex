%% USPSC-Resumo.tex
\setlength{\absparsep}{18pt} % ajusta o espaçamento dos parágrafos do resumo		
\begin{resumo}
	\begin{flushleft} 
			\setlength{\absparsep}{0pt} % ajusta o espaçamento da referência	
			\SingleSpacing 
			\imprimirautorabr~~\textbf{\imprimirtituloresumo}.	\imprimirdata. \pageref{LastPage}p. 
			%Substitua p. por f. quando utilizar oneside em \documentclass
			%\pageref{LastPage}f.
			\imprimirtipotrabalho~-~\imprimirinstituicao, \imprimirlocal, \imprimirdata. 
 	\end{flushleft}
	\OnehalfSpacing 			
	Este trabalho tem como propósito a aplicação de modelos de aprendizado de máquina e técnicas de inteligência artificial explicável aos dados da Coordenação de Internações de Ribeirão Preto, Brasil, com o objetivo de predizer reinternações hospitalares. A metodologia abrangeu o treinamento e seleção de classificadores em três fases distintas, culminando na análise de importância de variáveis por meio do método de explicação \textit{SHapley Additive exPlanations}. Os resultados indicam que pacientes solteiros, que ficaram internados por menos horas e são mais jovens, apresentam maior risco de reinternação. Variáveis sazonais, como o mês ou o dia da semana de internação também demonstraram grande influência na predição de reinternação. No entanto, as acurácias dos modelos permaneceram abaixo de 60\%, apontando para oportunidades de aprimoramento na performance preditiva. Alternativas foram sugeridas para futuras melhorias, como a expansão do conjunto de dados e a exploração de modelos mais complexos. Essas conclusões, além de orientar em melhorias específicas neste contexto, fornecem uma sólida base para pesquisas futuras, não apenas nesta base de dados específica, mas como diretrizes para investigações similares em outras bases, contribuindo para avanços contínuos na predição de reinternações hospitalares.
	
	\vspace{\onelineskip}
	\noindent
	\textbf{Palavras-chave}: Inteligência Artificial Explicável. Aprendizado de Máquina. Reinternações Hospitalares. Serviços de Saúde Mental.
\end{resumo}